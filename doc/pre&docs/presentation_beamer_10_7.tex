\documentclass[utf8, a4paper]{beamer}
\usepackage{CJK}
\usepackage[utf8]{inputenc}

\usepackage{amsmath, amsfonts, epsfig, xspace}
\usepackage{algorithm,algorithmic}
\usepackage{pstricks,pst-node}
\usepackage{multimedia}
\usepackage[normal,tight,center]{subfigure}
\setlength{\subfigcapskip}{-.5em}
\usepackage{beamerthemesplit}

\author[Software Engineering 2017, Tsinghua University]{SigmaGo}

\title[Competition \& Lecture Management System\hspace{2em}\insertframenumber/\inserttotalframenumber]{Competition \& Lecture Management System
}

\institute{清华大学}

\begin{document}

\begin{CJK}{UTF8}{gbsn}
\maketitle


\begin{frame}
    \frametitle{功能介绍\&需求分析}
  %\pause
    \begin{enumerate}
        \item \textbf{获取活动信息}
            \begin{itemize}
            \item 目前,比赛、讲座信息分散同学们没有方法获取。现在获取活动信息中缺少一些标准化的流程,比如赛事的进程和获奖等,又比如讲座的主办方、主讲人等。比赛推进的进程中也缺少跟进,现在只有推送说明,甚至部分主办方不发推送。
        
            \item 校内师生需要通过本平台获取最新的活动(包括比赛、讲座)信息,并方便地获取活动推进过程中各个环节的信息。
            \end{itemize}
            \pause
        \item \textbf{活动推荐}
            \begin{itemize}
            \item 一方面,校内师生需要根据个人的情况,获得活动的推荐;另一方面,校内师生还需要获得近期热门的活动的推荐。
            \end{itemize}
            
            \pause
        \item \textbf{获奖信息统计}
            \begin{itemize}
            \item 现代阶段比赛结果缺少一个统一的平台管理,无法对于获奖有效性进行验证。
            \end{itemize}
  \end{enumerate}
 
  
  
\end{frame}

\begin{frame}
    \frametitle{用户群体}
  
    \begin{itemize}
    \item \textbf{校内师生}
  
    访问量大,要求响应较快。
  
    \item \textbf{活动主办方}
  
    访问量小,要求严格的权限控制,但对响应速度要求很低。需要精确的统计以及友好的功能,使得主办方主动参与到本系统中。
  
    \item \textbf{管理员}
  
    访问量小,要求严格的权限控制,但对响应速度要求很低。为了降低维护成本,需要支持无编程基础的管理员。

    \end{itemize}
\end{frame}

\begin{frame}
    \frametitle{现有平台}
  
    \begin{itemize}
    %\item 第二成绩单
    \item 各个院系科协或相关组织的公众号
    \item info上文素讲座信息
    
    

    \end{itemize}
    \par
    \textbf{以上信息需要进一步整合,现阶段缺少一个统一的平台进行组织和管理}
\end{frame}

\end{CJK}
\end{document}
