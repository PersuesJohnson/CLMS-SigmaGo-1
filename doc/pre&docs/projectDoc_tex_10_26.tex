\documentclass[12pt]{article}
\usepackage[utf8]{inputenc}
\usepackage{CJK}
\usepackage{booktabs}  % professionally typeset tables
\usepackage{amsmath}
\usepackage{setspace}
\usepackage{textcomp}  % better copyright sign, among other things
\usepackage{xcolor}
\usepackage{lipsum}    % filler text
\usepackage{subfig}   
\usepackage{geometry}
\usepackage{graphicx}
\geometry{a4paper,scale=0.8}
\date{}
\title{\textbf{竞赛讲座管理系统软件过程和项目计划}}
\author{\rightline{SigmaGo小组}} 


\begin{document}
\begin{CJK}{UTF8}{gbsn}
\begin{spacing}{1.3}
\maketitle
%\indent\indent
针对于校园中讲座和竞赛资源的流通性不足、同学们获取信息和资源的方式过于分散等问题,我们小组希望能够设计一个竞赛讲座管理系统,通过对于竞赛、讲座等信息进行统一整理和总结,并且设计一种算法对于特定同学进行合适的讲座和竞赛的推荐,增加同学们对于自己所需要的讲座、感兴趣的竞赛的认知,从而实现信息的最大化利用。

\section{拟采用模型}
\indent \indent
针对于讲座及竞赛信息进行网站设计,实现对于用户的特定赛事推荐和管理方的赛事管理的双向功能的平台的开发,实现平台自动整合赛事主办方提供的信息,按照用户需求提供给用户的一个完整地系统前后端的设计。

\section{分工}
\indent \indent
本系统总体分为前端、后端和文档两个部分。其中因为个人对于个人所负责部分较为熟悉,因此本项目遵循谁写的代码就由谁来负责该模块的文档的编写工作这一原则,将文档部分的任务均匀分给全体成员。前端、后端的分工初步安排如下:


* 前端:刘昊天、曹逸宁负责;
    
* 后端:黄秀峰、韩益增、张知行、朱海东负责。


以上安排为目前暂定安排计划,之后具体分工在以上安排的基础上,根据具体的工作进展情况进行后期的微调。每人负责自己模块的文档编写工作。


\section{进度安排}
\indent \indent
本项目的进度安排大致如下:
\begin{enumerate}
    \item 教学周第五周前:完成当前所希望设计的网站的初步设计和规划,大体明确网站的功能、架构与逻辑部分,对于现有的系统情况进行调研,进行初步任务量估计,指定大体的系统设计的安排和计划。
    
    \item 教学周第六周前:完成初始版本的项目计划书的编写,进行初步的项目分工安排和计划,对于项目的细节化部分进行更进一步的设计安排。同时前端部分根据上一项所设计的计划安排进行初步的网站前端的编写工作,进行前端的框架搭建和项目编写;同时后端部分开始进行框架的搭建,明细化个人分工,初步尝试在前端的现有基础上进行规划和架构。
    
    \item 教学周第十周前:完成前端设计、安排与工作,后端部分大体逻辑层面完成,实现最初设计的网站的初始功能。同时开始对于信息推荐算法进行设计,通过合理的设计处理针对于不同的特定同学的赛事推荐的功能的初步实现。
    
    \item 教学周第十二周前:完成推荐算法的初步设计,对于推荐算法进行部署。同时对于后台管理员的管理平台进行进一步设计,通过设计出一个比较合理的有好的后台管理平台来使该系统能够更好进行后期维护和运营,方便赛事和讲座主办方进行信息的录入。
    
    \item 教学周第十四周前:完成平台的debug工作和平台的初步测试,对于平台可能存在的问题进行修正,并且增强平台的鲁棒性。
    
    \item (选)如果能够按时甚至提前实现这一系统项目计划安排,并且依旧能够有足够的时间进行其他的功能的开发的话,计划在在该平台上提供更多的功能,并且将设计除网站外的其他平台上面的该系统的实现,降低该平台的推广的难度。
\end{enumerate}


\section{版本管理计划}
\indent\indent
通过github统一进行管理,采用pr-审核制,每一块完成之后须有另外的人进行审核之后才能merge进入仓库中,一方面可以一定程度上在编写时就降低bug的产生的概率,另一方面可以让组中的成员相互间了解整个平台的运行结构,彼此熟悉对方的工作,有利于后续工作的进行。

\section{风险管理计划}
\indent\indent
\#其实我相信我们组的大腿们都是足够有责任心不会退课的233

如果项目存在完不成的风险,则在项目进度安排中完全不考虑选作部分,并且对于算法的设计等内容使用现有的轮子或直接使用比较简单的推荐方法(比如按照系别和比赛时间进行推荐等),运营平台的友好性和前端的友好性比预期部分进行适当的降低,将目标转换成实现大体的框架搭建和一个能够后期运行的平台的设计,而将之后的平台维护等工作内容后移到课程之后平台能够推广时,交由运营方维护或者让我们在后期有时间的话再进行维护。

%\section{实验收获与小结}
%\section{程序来源说明}
\end{spacing}
\end{CJK}
\end{document}
